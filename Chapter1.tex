\chapter{\large INTRODUCTION}
\thispagestyle{empty}
\setcounter{page}{1}
\pagenumbering{arabic}
% -------- begin content from here--------

This chapter introduces the content of this thesis.

For research proposal you should use future tense style in your reporting. For example you will state that ``I will formulate the system using formal language''.  For completed research work you should use reported tense  style in your reporting. For example you will state that ``I formulated the system using formal language''.  

You are expected to give clear explanation to the content of the thesis using simple sentences. You are expected to summaries the following in this introductory chapter.

\section{Background}
In this section you should document the background to this research in terms of the research question it answers. Provide the computing background to the philosophy, theory or metaphor underlying the research. Provide the context and scope of the research by which the statement of the research problem. Also summarise the following in clear explanation.

\section{Statement of the Project Problem}

The statement of research problem should state four things clearly: $(i)$ The state of the art in the literature.  $(ii)$ The gap or problem identified in documented literature. $(iii)$ Why is it important to address the problem? $(iv)$ How did you address the problem? 

This can take the following form. 

\vspace*{3mm}

\begin{tabular}{||p{14cm}||} \hline
Various approaches to the solution of problem of $X$ is well documented in the literature. However, the effectiveness of method $P$ has not been demonstrated on sparse data set. Most problems in human face recognition involves sparse data manipulation. This research applied method $P$ to sparse data of humans face with dark skins. \\ \hline 
\end{tabular}


\section{Aim and Objectives of Project}

State the aim. There should be one aim for your research. The aim is the final outcome of the research.  The objectives are the individual tasks in the aim.  The objective will state what you did at each task. For example it may include:

\begin{enumerate}
\item Specification of the model or the system. \vspace*{-4mm}
\item Data design and collection. \vspace*{-4mm}
\item Formulation of the model. \vspace*{-4mm}
\item Design of the model. \vspace*{-4mm}
\item Implementation of the model or system. \vspace*{-4mm}
\item Evaluation. 
\end{enumerate}

\section{Project Methodology}
In the methodology you are expected to document how you achieved each of the objective. You need to state the methods you used for each task and how your combined all the methods during the realisation of the research problem-solving. Specifically you are to state the methods you used to:
  
\begin{enumerate}
\item Specify the system of model: State the characteristics of the input and output of the model of system you have developed. Provide the requirement for the input and the output by imposing restriction on the scopes of values or strings accepted and acceptable. Your will need to state assumptions that informs your requirements. \vspace*{-4mm}

\item Formulate your system or model: Use symbols to represent the inputs and the output and state the process by which the output is computed from the input using the process. \vspace*{-4mm}

\item Design the system: E.g, Use design tools such as flow chart, UML diagrams, tree, entity relation diagram, sequence diagram, circuit diagram, timing diagram, etc. for the model or system. \vspace*{-4mm}

\item Implement the model or system. This may include program coding, simulation and/or hardware aspect \vspace*{-4mm}

\item Evaluate the system: This may include alpha-beta method, mean-opinion score, etc. 
\end{enumerate} 

\section{Operational definition of terms}

\section{Project Scope}

\section{Research Philosophy}

State the theories of computing that defines the framework for this work. This section is particularly very important for Ph.D. thesis.


\section{Research Justification}

Why do you think solving this problem is worth the effort. This is a general perspective.

\section{Contribution to Knowledge} 

This should arise from your research statement of problem. The unique attribute of the technique you apply to the problem should be stated here. 

\section{Organisation of Thesis}

State how this thesis is arranged from Chapter Two onwards.  