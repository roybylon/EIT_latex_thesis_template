\chapter{\large LITERATURE REVIEW}
\thispagestyle{empty}
% -------- begin content from here--------

\section{Fundamental Concepts}
In this section you are to present carefully selected fundamental concepts. They are fundamental because other concepts can be logically derived from them.  Identify and list all such concepts that are key to the explanation to your research discourse. To harvest such concepts, you need to consult  the literature as documented in encyclopedia, standard dictionary, journals, handbooks and general use in technical discourse. Then discuss the context in which you will be using these concepts in your thesis.


Note that \textbf{Chapter} heading should be in all upper case. \textbf{Section} should be in title case as depicted in the section. Note that the title fo \textbf{Subsection} and definition should be in sentence case as indicated in the following illustrations.


 
\subsection{Definition of Terms}


A simple one term definition should appear as depicted in Definition \ref{Def1}:

\begin{definition}[Computing]
Computing is the creation and manipulation of concepts in the process of constructing the solution to a well-defined problem.	
\label{Def1}
\end{definition}

%A theorem should appear as follows:
%
%\begin{theorem}[Computing]
%	Computing is the creation and manipulation of concepts in the process of constructing the solution to a well-defined problem.	
%\end{theorem}

A more elaborate definition should appear as depicted in Definition \ref{Def2}:

\begin{definition}[Grammar]
The grammar of a grammar $G$ is a four tuple $\mathbf{G = \left\langle \Sigma, V, P, S \right\rangle}$ where;
\begin{itemize}
\item $\Sigma$ is the finite, non-empty, set of symbols comprising the alphabet of the language of $G$;
\item $V$ is the finite, non-empty set of non-terminal symbols comprising the vocabulary of the language of $G$;
\item $P$ is the non-empty finite set of re-write rules or productions for generated strings for the expression of the language of $G$;
\item $S$ is a distinguish non-terminal start symbol.  
\end{itemize}
\label{Def2}
\end{definition}

		 

\section{Analysis of Theories}

Relevant, related and core theories and models
should be explained here. Your explanation should engage the discourse in the literature, not a mere listing or verbose presentation of contents. To do this, you should write out the questions that you think the paper you are reading should answer. While reading the paper, determine for yourself whether the paper has answered the questions adequately in you own view. Your questions and responses are what you should document in your theoretical analysis of the literature. PhD students are  required to comment on the theoretical foundations presented in the literature and situate their analysis in wider context and perspective by suggesting important, and perhaps new, lines of argument based on present circumstance.

\section{Technical Foundations}

Discuss the Techniques, methods, design, technologies, model and tools in the literature that are relevant to this research.

Table \ref{Table1} depicts how to include basic table into your text. You must discuss each and very illustration in the text. To do this state what each column of item contains ans discuss at least two cases of the row items in the table. 

\begin{table}[h!]
\caption{Items in Research Steps}
\begin{tabular}{p{3.0cm} p{10.0cm}} \hline \hline
Item  & Description \\ \hline \hline	
Data collection & Observation, questionnaire, digital camera, thermometer, transducers, video recorder, microphone, etc;  \\ \hline
Data encoding & Formatting (e.g. field definition), code definition (e.g. Binary coded), key determination, indexing procedure, database structuring (linear or non-linear);  \\ \hline
Specification & Formal language, UML use case diagram \\\hline
Formulation & Equation, model digram, formal method language, automata model formula \\\hline
Design & Flowchart, UML diagrams, logic diagrams, state transition diagrams and tables; schema, \\\hline
Simulation  & Simulation software e.g. CNP tool, JFLAP \\\hline
Implementation & Programming language, software and hardware development tools\\ \hline
Evaluation & Testing of simulated or implemented software and hardware\\ \hline \hline
\end{tabular}
\label{Table1}
\end{table}

 
\section{Review of Related Work}
Engage the literature from the theoretical perspective you have selected (PhD candidate must do this). Your engagement should be with the view to explain and discuss the issues presented and documented in the literature, not a mere listing or verbose presentation of contents. To do this, you should write out the questions that you think each of the paper you read should answer. While reading the paper, determine for yourself whether the paper has answered the questions adequately in you own view. Your questions and responses is what you should document in your analysis of the literature.

Focus more on the problem being solved by the author. The method used to state specification, collect data, formulate models, design model, implement system, evaluate system as well as analysis of results. Explain how the concepts presented in the paper is useful or not adequate for your work.

A literature review table will be very useful for this purpose. Create a literature review table as depicted in Table \ref{LRTable1} in Appendix \textbf{D}. 


This is the format for citing sources. 
To cite book use \cite{Hollis:1999:VBD:519964} \cite{Dreyfus92} for one or two authors and \cite{Goossens:1999:LWC:553897} for more than two authors. Technical report should cited as \cite{897367}. Conference inproceedings should be cited as \cite{Geach68,Fulga2012}. 
Thesis should be cited as \cite{Clarkson:1985:ACP:911891, Olorunfemi2018} for Ph.D and Master thesis as \cite{Dingemanse2006,Oladimaji2012}.
Article in journal should be cited as \cite{SaeediMEJ10} \cite{McCarthy07, LawalTwoness, TurningPaper1950} . Work in-collection, should be cited as  \cite{Li-2008}. Miscellaneous source, for example personal communication should be cited as \cite{Olorode2016}. Booklet such as inaugural lecture should be cited as \cite{Akiwowo:Ajobi1980}. Your can use out-text citing as \citep{McCarthy07}.

How these citations should be listed is presented in the Reference section of this document.

\section{Chapter Summary}
Provide a summary for this chapter.

Discuss the six(6) literature items that are   core to this research and state how they will inform the research activity and its methodology. 
