\chapter{\large RESULT AND DISCUSSION}
\thispagestyle{empty}
% ---
\section{Discussion of Results}
Analyse and discuss the results you obtained as precisely as possible. 


Result that include range of values should appear as depicted in 
Table \ref{tab-rang}.

\begin{table}[!h]
	\caption{CPU Performance During Simulation}
	\label{tab-rang}
	\centering
	\begin{tabular}{ccl}
		\hline \hline
		Tests cycle & CPU utilisation (\%) & Average response time (Sec)\\
		\hline
		1. & 30 -- 40 & 6.0 -- 8.0\\
		2. & 40 -- 50 & 5.0 -- 7.0\\
		3. & 50 -- 60 & 3.0 -- 5.0\\
		4. & 60 -- 70 & 2.0 -- 4.0\\
		5. & 70 -- 80 & 1.0 -- 3.0\\
		\hline\hline 
	\end{tabular}
\end{table}



\section{Research Findings}
Explain the interesting aspect of your findings and relate them with other findings in the literature. For PhD thesis, how the theory explains the result must be stated and demonstrated with incisive examples. Cases of problem that cannot be handle by the research solution approach must also be explained from a theoretical perspective. 

 
\section{Implication of Findings}
Discuss the possible interpretation of your findings and applications to users or other researches. For example, the usefulness of your data in other researches; if so how and how do you intend to make it available to other researcher.
Can your model be used to develop a working system, if so how will this be realised together with possible cost of deployment. What utility will the outcome of this research give to practical and theoretical activities in computing?

\section{Chapter Summary}

Summary of the contents of is chapter.
